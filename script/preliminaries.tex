\section{Preliminaries}
\label{sec:preliminaries}

After the informal introduction of the example program in the last chapter, this chapter will introduce formal definitions.
In the first part there will be presented a formal definition of program graphs.
In the second part different types of invariants, which are necessary for the analysis, will be introduced.

\subsection{Program}

We define $\mathcal{L} = \lbrace \ell_0, \dots, \ell_n \rbrace$ as the set of program locations, where $\ell_0$ denotes the unique start location of a program.
The example program uses the location set $\mathcal{L} = \lbrace \ell_0 , \ell_1 , \ell_2 \rbrace$ and $\ell_0$ is its start location.

We define $\mathcal{V} = \lbrace v_1, \dots, v_n \rbrace$ as the set of variables which occur in a program.
We define $\mathcal{V}' = \lbrace v'_1, \dots, v'_n \rbrace$ as annotated variables to distinguish between previous and later values.
In the example program there occur the variables $x$, $y$ and $i$.
Their annotated versions are $x'$, $y'$ and $i'$.

An evaluation $\upsilon: \mathcal{V} \rightarrow \mathbb{Z}$ assigns each program variable an integer value.
A state $s = (\ell, \upsilon)$ represents the values of variables at a certain location $\ell \in \mathcal{L}$.
A valid starting state in the example program could be for instance $(\ell_0, \upsilon_{[x \rightarrow 1, y \rightarrow 1, i \rightarrow 1]})$

We denote with $\mathcal{F}(\mathcal{V})$ the set of formulas consisting of conjunctions of linear inequalities over the variables $\mathcal{V}$.
We define a program as a directed graph consisting of the program locations $\mathcal{L}$ and a set of transitions $\mathcal{T} = \lbrace (\ell, \tau, \ell') \mid \ell, \ell' \in \mathcal{L}, \tau \in \mathcal{F}(\mathcal{V} \cup \mathcal{V}') \rbrace$ between those locations.
If for a variable $v$ no assignment $v' = \dots$ is given, we assume the trivial assignment $v' = v$.
The example program consists of seven transitions. The formal representation of $t_3$ is for instance $(\ell_1, i > 0 \wedge x' = x + 1 \wedge i' = i - 1 \wedge y' = y,\ell_1)$.

An evaluation step with a transition $(\ell, \tau, \ell') \in \mathcal{T}$ leads a program state $(\ell, \upsilon)$ to another program state $(\ell', \upsilon')$ if and only if $\upsilon \models \tau$ and $\upsilon' \models \tau$. We then write $(\ell, \upsilon) \rightarrow_t (\ell', \upsilon')$.
For instance $(\ell_1,\upsilon_{[i \rightarrow 0, x \rightarrow 1, y \rightarrow 2]}) \rightarrow_{t_4} (\ell_2,\upsilon_{[i \rightarrow 0, x \rightarrow 1, y \rightarrow 2]})$ is a valid evaluation step in the program, while $(\ell_1,\upsilon_{[i \rightarrow 1, x \rightarrow 1, y \rightarrow 2]}) \rightarrow_{t_4} (\ell_2,\upsilon_{[i \rightarrow 1, x \rightarrow 1, y \rightarrow 2]})$ is not a valid evaluation step, since $i = 0$ does not hold for both evaluations.

We define a program component $\mathcal{C} \subseteq \mathcal{T}$ as a strongly connected component (SCC) of a program.
We denote with $\mathcal{E}_\mathcal{C}$ its entry transition set which consists of all transitions $t = (\ell, \tau, \ell')$ such that $\ell'$ appears in $\mathcal{C}$ but $t \notin \mathcal{C}$.

The example program contains three SCCs. Every location represents in this case its own SCC. On the one hand there is the trivial SCC $C_0 = \emptyset$ for the location $\ell_0$, on the other hand there are the nontrivial SCCs $C_1 = \lbrace t_3 \rbrace$ and $C_2 = \lbrace t_5 \rbrace$.
The entry transitions of those SCCs are $\mathcal{E}_{\mathcal{C}_0} = \lbrace t_0 \rbrace$, $\mathcal{E}_{\mathcal{C}_1} = \lbrace t_1 \rbrace$ and $\mathcal{E}_{\mathcal{C}_2} = \lbrace t_2, t_4 \rbrace$.
Figure \ref{fig:sccs} shows the three SCCs and their entry transitions.

\begin{figure}
\centering
\begin{tikzpicture}[->,>=stealth',auto,node distance=3.5cm,
    thick,
    main node/.style={circle,draw,font=\sffamily\Large\bfseries},
    aligned edge/.style={align=left}]

  \node[main node, very thick, densely dotted] (0) {$l_0$};
  \node[main node, very thick, solid] (1) [right of=0] {$l_1$};
  \node[main node, densely dashed, very thick] (2) [right of=1] {$l_2$};

  \node (4) [right of=2] {};
  \node (5) [left of=0] {};

  \path[every node/.style={font=\sffamily\small}]
    (0) edge[solid, thin] node {$t_1$} (1)
        edge[loosely dashed, thin, bend right=25] node [below] {$t_2$} (2)
    (1) edge[solid, very thick, loop above] node {$t_3$} (1)
        edge[loosely dashed, thin] node {$t_4$} (2)
    (2) edge[densely dashed, very thick, loop above] node {$t_5$} (2)
        edge node {$t_6$} (4)
    (5) edge[loosely dotted] node {$t_0$} (0);
\end{tikzpicture}
\caption{The example program with the SCCs and entry transitions marked}
\label{fig:sccs}
\end{figure}


We define an assertion as a pair $(t, \phi)$ where $t \in T$ and $\phi \in \mathcal{F}(\mathcal{V})$.
We say that a program is safe for an assertion if and only if $\forall (\ell_0, \upsilon_0): (\ell_0, \upsilon_0) \rightarrow_P^* \circ \rightarrow_t (\ell, \upsilon) \Rightarrow \upsilon \models \phi$.
We say that a program is conditional safe for an assertion if from a state where a precondition holds every path to an assertion also leads to the fulfillment of the assertion.
The example program contains one assertion $(t_6, x \neq y)$. The goal is to show that the program is safe for this assertion. The algorithm uses the conditional safety property for this purpose.

\subsection{Invariants}

An invariant is defined as an assignment $\mathcal{I} : \mathcal{L} \rightarrow \mathcal{F}(\mathcal{V})$ if and only if for all reachable states $(\ell, \upsilon)$ it holds that $\upsilon \models \mathcal{I}(\ell)$.
For instance a function with $\mathcal{I}(\ell) = x > 0$ for each location $\ell \in \mathcal{L}$ is an invariant.

An invariant is inductive if and only if additionally it holds that $\top \models \mathcal{I}(\ell_0)$ and for all $(\ell, \tau, \ell') \in \mathcal{P}$ it holds that $\mathcal{I}(\ell) \wedge \tau \models \mathcal{I}(\ell')'$. We call the first initiation condition and the latter consecution condition.
$\mathcal{I}$ with $\mathcal{I}(\ell_0) = i > 0$, $\mathcal{I}(\ell_1) = x > y \vee i > 0$ and $\mathcal{I}(\ell_2) = x \neq y$ is an inductive invariant of the example program as we will see.

An invariant is an conditional inductive invariant if and only if for all $(\ell, \upsilon) \rightarrow_\mathcal{P} (\ell', \upsilon')$ it holds that $\upsilon \models \mathcal{Q}(\ell) \Rightarrow \upsilon' \models \mathcal{Q}(\ell')$.
In contrast to inductive invariants those conditional inductive invariants does not hold from the program start on, but from a specific state on.
We call conditional inductive invariants CII from this point on.
