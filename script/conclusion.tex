\section{Conclusion}
\label{sec:conclusion}

In this paper we explained the method of compositional safety verification with an example program.
This example program covered many distinctions of cases and therefore enabled us to take a deeper look at the inner workings of the algorithm.
We inspected several things left out in the original paper:
\begin{enumerate}
\item The exploration of multiple SCCs occurring before an SCC inspected by CheckSafe
\item The effect of narrowing if a path to the SCC is proved safe and another could not proved safe
\item The behavior of CheckSafe if a CII with multiple conjuncts is found
\end{enumerate}

On the other hand there are cases which could be further investigated in subsequent studies.
We focused on SCCs with only one transition.
This affects the search for CIIs and narrowing.
For the search for CIIs inspecting multiple transitions in an SCC has just the effect of giving further consecution constraints to the Max-SMT-Solver.
For narrowing it affects both the added constraints in the SCC as well as the added constraints to the entry transitions, if they lead to different entry locations.
Further studies could therefore inspect SCCs with multiple transitions to give a fine-grained explanation regarding narrowing.
