\section{Introduction}
\label{sec:introduction}

\subsection{Compositional safety verification}

Safety verification is the task to ensure that an assertion at a certain program location is always true regardless of the input.
Formally a program is safe for an assertion if every evaluation path of the program leads to the fulfillment of the assertion.
While this definition describes a global analysis of the whole program, it is also possible to analyze components of a program and combine the results.
The advantage of this compositional method is its scalability, but in many cases there is a loss in precision due to fewer information in the separated analysis.

Since variables in a program can potentially be global and therefore a change of a variable at a location can have a huge impact on other parts of the program, it is not possible to analyze parts of the program completely independent. \todo{Example?}
Therefore the presented algorithm analyses parts semi-independently, where the analysis of the parts gets combined and eventually reanalyzed.

\subsection{Structure}

This paper aims at explaining the algorithm for compositional safety verification with a single example.
In the next section of this chapter this example will be presented.
The following chapter will cover formal definitions of program graphs and invariants and illustrate them by the presented example.

The third chapter will introduce the algorithm. 
At first an overview over the basic analysis of a whole program will be given and explained by the example, after that narrowing and the analysis of components will be explained.
For the analysis of components Max-SMT-Solving, an extension of SMT-Solving, will be introduced.

In the end a summary of the method will be given.

\subsection{Example program}

\begin{figure}
\centering
\begin{tikzpicture}[->,>=stealth',auto,node distance=3.5cm,
    thick,
    main node/.style={circle,draw,font=\sffamily\Large\bfseries},
    aligned edge/.style={align=left}]

  \node[main node] (0) {$l_0$};
  \node[main node] (1) [right of=0] {$l_1$};
  \node[main node] (2) [right of=1] {$l_2$};

  \node (4) [right of=2] {};
  \node (5) [left of=0] {};

  \path[every node/.style={font=\sffamily\small}]
    (0) edge[aligned edge] node {$t_1: x \geq y$} (1)
        edge[aligned edge, bend right=25] node [below] {$t_2: y > x$} (2)
    (1) edge[aligned edge, loop above] node {$t_3:$\\$i > 0 \wedge$\\$x' = x + 1 \wedge$\\$i' = i - 1$} (1)
        edge[aligned edge] node {$t_4: i = 0$} (2)
    (2) edge[aligned edge, loop above] node {$t_5:$\\$x' = x + 1 \wedge$\\$y' = y + 1$} (2)
        edge[aligned edge] node {$t_6$, assert($x \neq y$)} (4)
    (5) edge[aligned edge] node {$t_0: i > 0$} (0);
\end{tikzpicture}
\caption{Example program}
\label{fig:example}
\end{figure}

To show the inner workings of the algorithm, we will use a single example and explain based on that the different techniques of the algorithm.

The example program is shown in \ref{fig:example}.
It has three input parameters $x>0$, $y>0$ and $i>0$.
The task of the safety verification is to assure, that after every possible evaluation $x \neq y$ holds.

We can easily show this property of the program intentionally:
Starting at $l_0$ we can move on to $l_2$ if $y>x$.
With every iteration with $t_5$ also $y+1>x+1$ holds and therefore $x \neq y$ holds in the end of the program.
Starting at $l_0$ we can also move to $l_1$ if $x \geq y$.
Since $i>0$ we must use $t_3$ at least one time before moving on to $l_2$.
Since $t_3$ increments the value of $x$, $x > y$ holds after one usage of $t_3$.
Again $t_5$ can not change the relation between $x$ and $y$ and therefore $x \neq y$ holds also for this case.

\todo{Put whole block after program definition?}
